\documentclass[]{article}
\usepackage{lmodern}
\usepackage{amssymb,amsmath}
\usepackage{ifxetex,ifluatex}
\usepackage{fixltx2e} % provides \textsubscript
\ifnum 0\ifxetex 1\fi\ifluatex 1\fi=0 % if pdftex
  \usepackage[T1]{fontenc}
  \usepackage[utf8]{inputenc}
\else % if luatex or xelatex
  \ifxetex
    \usepackage{mathspec}
  \else
    \usepackage{fontspec}
  \fi
  \defaultfontfeatures{Ligatures=TeX,Scale=MatchLowercase}
\fi
% use upquote if available, for straight quotes in verbatim environments
\IfFileExists{upquote.sty}{\usepackage{upquote}}{}
% use microtype if available
\IfFileExists{microtype.sty}{%
\usepackage{microtype}
\UseMicrotypeSet[protrusion]{basicmath} % disable protrusion for tt fonts
}{}
\usepackage[margin=1in]{geometry}
\usepackage{hyperref}
\hypersetup{unicode=true,
            pdftitle={Time Series Final},
            pdfauthor={Steve Bramhall},
            pdfborder={0 0 0},
            breaklinks=true}
\urlstyle{same}  % don't use monospace font for urls
\usepackage{color}
\usepackage{fancyvrb}
\newcommand{\VerbBar}{|}
\newcommand{\VERB}{\Verb[commandchars=\\\{\}]}
\DefineVerbatimEnvironment{Highlighting}{Verbatim}{commandchars=\\\{\}}
% Add ',fontsize=\small' for more characters per line
\usepackage{framed}
\definecolor{shadecolor}{RGB}{248,248,248}
\newenvironment{Shaded}{\begin{snugshade}}{\end{snugshade}}
\newcommand{\KeywordTok}[1]{\textcolor[rgb]{0.13,0.29,0.53}{\textbf{#1}}}
\newcommand{\DataTypeTok}[1]{\textcolor[rgb]{0.13,0.29,0.53}{#1}}
\newcommand{\DecValTok}[1]{\textcolor[rgb]{0.00,0.00,0.81}{#1}}
\newcommand{\BaseNTok}[1]{\textcolor[rgb]{0.00,0.00,0.81}{#1}}
\newcommand{\FloatTok}[1]{\textcolor[rgb]{0.00,0.00,0.81}{#1}}
\newcommand{\ConstantTok}[1]{\textcolor[rgb]{0.00,0.00,0.00}{#1}}
\newcommand{\CharTok}[1]{\textcolor[rgb]{0.31,0.60,0.02}{#1}}
\newcommand{\SpecialCharTok}[1]{\textcolor[rgb]{0.00,0.00,0.00}{#1}}
\newcommand{\StringTok}[1]{\textcolor[rgb]{0.31,0.60,0.02}{#1}}
\newcommand{\VerbatimStringTok}[1]{\textcolor[rgb]{0.31,0.60,0.02}{#1}}
\newcommand{\SpecialStringTok}[1]{\textcolor[rgb]{0.31,0.60,0.02}{#1}}
\newcommand{\ImportTok}[1]{#1}
\newcommand{\CommentTok}[1]{\textcolor[rgb]{0.56,0.35,0.01}{\textit{#1}}}
\newcommand{\DocumentationTok}[1]{\textcolor[rgb]{0.56,0.35,0.01}{\textbf{\textit{#1}}}}
\newcommand{\AnnotationTok}[1]{\textcolor[rgb]{0.56,0.35,0.01}{\textbf{\textit{#1}}}}
\newcommand{\CommentVarTok}[1]{\textcolor[rgb]{0.56,0.35,0.01}{\textbf{\textit{#1}}}}
\newcommand{\OtherTok}[1]{\textcolor[rgb]{0.56,0.35,0.01}{#1}}
\newcommand{\FunctionTok}[1]{\textcolor[rgb]{0.00,0.00,0.00}{#1}}
\newcommand{\VariableTok}[1]{\textcolor[rgb]{0.00,0.00,0.00}{#1}}
\newcommand{\ControlFlowTok}[1]{\textcolor[rgb]{0.13,0.29,0.53}{\textbf{#1}}}
\newcommand{\OperatorTok}[1]{\textcolor[rgb]{0.81,0.36,0.00}{\textbf{#1}}}
\newcommand{\BuiltInTok}[1]{#1}
\newcommand{\ExtensionTok}[1]{#1}
\newcommand{\PreprocessorTok}[1]{\textcolor[rgb]{0.56,0.35,0.01}{\textit{#1}}}
\newcommand{\AttributeTok}[1]{\textcolor[rgb]{0.77,0.63,0.00}{#1}}
\newcommand{\RegionMarkerTok}[1]{#1}
\newcommand{\InformationTok}[1]{\textcolor[rgb]{0.56,0.35,0.01}{\textbf{\textit{#1}}}}
\newcommand{\WarningTok}[1]{\textcolor[rgb]{0.56,0.35,0.01}{\textbf{\textit{#1}}}}
\newcommand{\AlertTok}[1]{\textcolor[rgb]{0.94,0.16,0.16}{#1}}
\newcommand{\ErrorTok}[1]{\textcolor[rgb]{0.64,0.00,0.00}{\textbf{#1}}}
\newcommand{\NormalTok}[1]{#1}
\usepackage{graphicx,grffile}
\makeatletter
\def\maxwidth{\ifdim\Gin@nat@width>\linewidth\linewidth\else\Gin@nat@width\fi}
\def\maxheight{\ifdim\Gin@nat@height>\textheight\textheight\else\Gin@nat@height\fi}
\makeatother
% Scale images if necessary, so that they will not overflow the page
% margins by default, and it is still possible to overwrite the defaults
% using explicit options in \includegraphics[width, height, ...]{}
\setkeys{Gin}{width=\maxwidth,height=\maxheight,keepaspectratio}
\IfFileExists{parskip.sty}{%
\usepackage{parskip}
}{% else
\setlength{\parindent}{0pt}
\setlength{\parskip}{6pt plus 2pt minus 1pt}
}
\setlength{\emergencystretch}{3em}  % prevent overfull lines
\providecommand{\tightlist}{%
  \setlength{\itemsep}{0pt}\setlength{\parskip}{0pt}}
\setcounter{secnumdepth}{0}
% Redefines (sub)paragraphs to behave more like sections
\ifx\paragraph\undefined\else
\let\oldparagraph\paragraph
\renewcommand{\paragraph}[1]{\oldparagraph{#1}\mbox{}}
\fi
\ifx\subparagraph\undefined\else
\let\oldsubparagraph\subparagraph
\renewcommand{\subparagraph}[1]{\oldsubparagraph{#1}\mbox{}}
\fi

%%% Use protect on footnotes to avoid problems with footnotes in titles
\let\rmarkdownfootnote\footnote%
\def\footnote{\protect\rmarkdownfootnote}

%%% Change title format to be more compact
\usepackage{titling}

% Create subtitle command for use in maketitle
\providecommand{\subtitle}[1]{
  \posttitle{
    \begin{center}\large#1\end{center}
    }
}

\setlength{\droptitle}{-2em}

  \title{Time Series Final}
    \pretitle{\vspace{\droptitle}\centering\huge}
  \posttitle{\par}
    \author{Steve Bramhall}
    \preauthor{\centering\large\emph}
  \postauthor{\par}
      \predate{\centering\large\emph}
  \postdate{\par}
    \date{August 13, 2019}


\begin{document}
\maketitle

\begin{Shaded}
\begin{Highlighting}[]
\KeywordTok{data}\NormalTok{(lap)}
\NormalTok{mort=}\KeywordTok{data.frame}\NormalTok{(}\DataTypeTok{date=}\KeywordTok{time}\NormalTok{(lap),}\DataTypeTok{Time=}\KeywordTok{as.factor}\NormalTok{(}\KeywordTok{seq}\NormalTok{(}\DecValTok{1}\NormalTok{,}\DecValTok{508}\NormalTok{,}\DecValTok{1}\NormalTok{)),}\KeywordTok{as.matrix}\NormalTok{(lap))}

\KeywordTok{library}\NormalTok{(psych)}
\end{Highlighting}
\end{Shaded}

\begin{verbatim}
## 
## Attaching package: 'psych'
\end{verbatim}

\begin{verbatim}
## The following objects are masked from 'package:ggplot2':
## 
##     %+%, alpha
\end{verbatim}

\begin{Shaded}
\begin{Highlighting}[]
\KeywordTok{pairs.panels}\NormalTok{(mort[,}\OperatorTok{-}\DecValTok{5}\NormalTok{], }
             \DataTypeTok{method =} \StringTok{"pearson"}\NormalTok{, }\CommentTok{# correlation method}
             \DataTypeTok{hist.col =} \StringTok{"#00AFBB"}\NormalTok{,}
             \DataTypeTok{density =} \OtherTok{TRUE}\NormalTok{,  }\CommentTok{# show density plots}
             \DataTypeTok{ellipses =} \OtherTok{TRUE} \CommentTok{# show correlation ellipses}
\NormalTok{             )}
\end{Highlighting}
\end{Shaded}

\includegraphics{SBramhall_Final_files/figure-latex/unnamed-chunk-1-1.pdf}

\begin{enumerate}
\def\labelenumi{\arabic{enumi}.}
\tightlist
\item
  Plot the respiratory mortality data.
\end{enumerate}

\begin{Shaded}
\begin{Highlighting}[]

\end{Highlighting}
\end{Shaded}

\begin{enumerate}
\def\labelenumi{\arabic{enumi}.}
\tightlist
\item
  Plot the respiratory mortality data.
\end{enumerate}

\begin{Shaded}
\begin{Highlighting}[]
\NormalTok{data =}\StringTok{ }\KeywordTok{read.csv}\NormalTok{(}\StringTok{"la_cmort_study.csv"}\NormalTok{, }\DataTypeTok{header=}\NormalTok{T)}

\CommentTok{# plot the temp}
\KeywordTok{plotts.sample.wge}\NormalTok{(data}\OperatorTok{$}\NormalTok{temp)}
\end{Highlighting}
\end{Shaded}

\includegraphics{SBramhall_Final_files/figure-latex/Prob 1-1.pdf}

\begin{Shaded}
\begin{Highlighting}[]
\KeywordTok{acf}\NormalTok{(data}\OperatorTok{$}\NormalTok{temp[}\DecValTok{1}\OperatorTok{:}\DecValTok{254}\NormalTok{])    }\CommentTok{# acf stationarity check}
\KeywordTok{acf}\NormalTok{(data}\OperatorTok{$}\NormalTok{temp[}\DecValTok{255}\OperatorTok{:}\DecValTok{508}\NormalTok{])  }\CommentTok{# acf stationarity check}

\CommentTok{# plot the pollution}
\KeywordTok{plotts.sample.wge}\NormalTok{(data}\OperatorTok{$}\NormalTok{part)}
\end{Highlighting}
\end{Shaded}

\includegraphics{SBramhall_Final_files/figure-latex/Prob 1-2.pdf}
\includegraphics{SBramhall_Final_files/figure-latex/Prob 1-3.pdf}

\begin{Shaded}
\begin{Highlighting}[]
\KeywordTok{acf}\NormalTok{(data}\OperatorTok{$}\NormalTok{part[}\DecValTok{1}\OperatorTok{:}\DecValTok{254}\NormalTok{])    }\CommentTok{# acf stationarity check}
\KeywordTok{acf}\NormalTok{(data}\OperatorTok{$}\NormalTok{part[}\DecValTok{255}\OperatorTok{:}\DecValTok{508}\NormalTok{])  }\CommentTok{# acf stationarity check}

\CommentTok{# plot the cardica mortality}
\KeywordTok{plotts.sample.wge}\NormalTok{(data}\OperatorTok{$}\NormalTok{cmort)}
\end{Highlighting}
\end{Shaded}

\includegraphics{SBramhall_Final_files/figure-latex/Prob 1-4.pdf}
\includegraphics{SBramhall_Final_files/figure-latex/Prob 1-5.pdf}

\begin{Shaded}
\begin{Highlighting}[]
\KeywordTok{acf}\NormalTok{(data}\OperatorTok{$}\NormalTok{cmort[}\DecValTok{1}\OperatorTok{:}\DecValTok{254}\NormalTok{])   }\CommentTok{# acf stationarity check}
\KeywordTok{acf}\NormalTok{(data}\OperatorTok{$}\NormalTok{cmort[}\DecValTok{255}\OperatorTok{:}\DecValTok{508}\NormalTok{]) }\CommentTok{# acf stationarity check}
\end{Highlighting}
\end{Shaded}

\includegraphics{SBramhall_Final_files/figure-latex/Prob 1-6.pdf}

\begin{Shaded}
\begin{Highlighting}[]
\KeywordTok{ggpairs}\NormalTok{(data[}\DecValTok{2}\OperatorTok{:}\DecValTok{4}\NormalTok{]) }\CommentTok{#matrix of scatter plots}
\end{Highlighting}
\end{Shaded}

\includegraphics{SBramhall_Final_files/figure-latex/Check correlation between variables-1.pdf}

 The variables appear independent so we will proceed with a univariate
analysis followed by Multiple Linear Regression (MLR) with correlated
errors.

\subsubsection{2. Comment on stationarity or
nonstationarity.}\label{comment-on-stationarity-or-nonstationarity.}

 The realizations for mortality, particulates, and temperature have some
pseudo cyclic behavior with multiple frequencies which is an indicator
of correlation. Their means and variances appear constant over time.
Their lags do not appear to be dependent on time as indicated by their
divided ACFs. There may be some slight wandering with the cardiac
mortality realization but this is one realization so the evidence
suggests stationarity.

\subsubsection{3a. Perform a univariate analysis using AR, ARMA, ARMIA
or ARUMA. Clearly explain how you arrived at your final model. Build a
neural network based model. Build an ensemble model between the two
models.}\label{a.-perform-a-univariate-analysis-using-ar-arma-armia-or-aruma.-clearly-explain-how-you-arrived-at-your-final-model.-build-a-neural-network-based-model.-build-an-ensemble-model-between-the-two-models.}

\begin{Shaded}
\begin{Highlighting}[]
\CommentTok{# -- Univariate Analysis of Particulates Using ARUMA --}

\NormalTok{part52 =}\StringTok{ }\KeywordTok{artrans.wge}\NormalTok{(data}\OperatorTok{$}\NormalTok{part,}\KeywordTok{c}\NormalTok{(}\KeywordTok{rep}\NormalTok{(}\DecValTok{0}\NormalTok{,}\DecValTok{51}\NormalTok{),}\DecValTok{1}\NormalTok{))      }\CommentTok{# since the data is weekly, remove weekly trend}
\end{Highlighting}
\end{Shaded}

\includegraphics{SBramhall_Final_files/figure-latex/Prob 3a-Univaritate Analysis for Particulates-1.pdf}

\begin{Shaded}
\begin{Highlighting}[]
\KeywordTok{acf}\NormalTok{(part52)                                         }\CommentTok{# confirm ACF is ~ white noise, it is!}

\CommentTok{# Perform model selection}
\KeywordTok{aic5.wge}\NormalTok{(part52)                                    }\CommentTok{# AIC picks ARMA(2,1)}
\end{Highlighting}
\end{Shaded}

\begin{verbatim}
## ---------WORKING... PLEASE WAIT... 
## 
## 
## Five Smallest Values of  aic
\end{verbatim}

\begin{verbatim}
##       p    q        aic
## 8     2    1   5.225047
## 6     1    2   5.227023
## 11    3    1   5.228819
## 14    4    1   5.230602
## 12    3    2   5.230981
\end{verbatim}

\begin{Shaded}
\begin{Highlighting}[]
\KeywordTok{aic5.wge}\NormalTok{(part52,}\DataTypeTok{type =} \StringTok{"bic"}\NormalTok{)                       }\CommentTok{# BIC picks ARMA(2,1)}
\end{Highlighting}
\end{Shaded}

\begin{verbatim}
## ---------WORKING... PLEASE WAIT... 
## 
## 
## Five Smallest Values of  bic
\end{verbatim}

\begin{verbatim}
##      p    q        bic
## 8    2    1   5.261210
## 1    0    0   5.261226
## 6    1    2   5.263185
## 4    1    0   5.271478
## 2    0    1   5.272038
\end{verbatim}

\begin{Shaded}
\begin{Highlighting}[]
\CommentTok{# Check for white noise}
\KeywordTok{ljung.wge}\NormalTok{(part52)                                   }\CommentTok{# FTR, p-value=.11}
\end{Highlighting}
\end{Shaded}

\begin{verbatim}
## Obs -0.05577662 0.1034012 0.05987676 0.1010256 0.05092448 0.09362802 0.09571357 0.08498643 0.0006229897 0.0780049 0.0384301 -0.0647129 0.02773225 -0.01937532 0.01298579 6.928832e-05 -0.009966296 0.03605867 0.01611233 0.01635121 -0.05353086 -0.04577465 0.01134892 0.0299041
\end{verbatim}

\begin{verbatim}
## $test
## [1] "Ljung-Box test"
## 
## $K
## [1] 24
## 
## $chi.square
## [1] 35.54288
## 
## $df
## [1] 24
## 
## $pval
## [1] 0.06074682
\end{verbatim}

\begin{Shaded}
\begin{Highlighting}[]
\KeywordTok{ljung.wge}\NormalTok{(part52, }\DataTypeTok{K =} \DecValTok{48}\NormalTok{)                           }\CommentTok{# FTR, p-value=.302}
\end{Highlighting}
\end{Shaded}

\begin{verbatim}
## Obs -0.05577662 0.1034012 0.05987676 0.1010256 0.05092448 0.09362802 0.09571357 0.08498643 0.0006229897 0.0780049 0.0384301 -0.0647129 0.02773225 -0.01937532 0.01298579 6.928832e-05 -0.009966296 0.03605867 0.01611233 0.01635121 -0.05353086 -0.04577465 0.01134892 0.0299041 0.004789332 -0.0009208109 0.05183014 -0.0660468 -0.05543925 0.02844053 0.01064092 0.007583799 -0.05369395 0.02009275 -0.003206871 -0.03212606 -0.004742175 0.04009265 -0.06741879 0.0864646 -0.05759432 -0.002036439 0.0327317 0.0237973 0.03137513 -0.04727448 -0.03951558 -0.02110117
\end{verbatim}

\begin{verbatim}
## $test
## [1] "Ljung-Box test"
## 
## $K
## [1] 48
## 
## $chi.square
## [1] 55.12436
## 
## $df
## [1] 48
## 
## $pval
## [1] 0.2232435
\end{verbatim}

\begin{Shaded}
\begin{Highlighting}[]
\KeywordTok{acf}\NormalTok{(part52,}\DataTypeTok{lag.max =} \DecValTok{48}\NormalTok{)                            }\CommentTok{# Box-Jenkins, acf looks consistent with white noise as previosly shown}

\CommentTok{# Plot ARUMA forecast, using ARUMA due to known weekly seasonality of data}
\NormalTok{predsPart =}\StringTok{ }\KeywordTok{fore.aruma.wge}\NormalTok{(data}\OperatorTok{$}\NormalTok{part,}\DataTypeTok{s =} \DecValTok{52}\NormalTok{, }\DataTypeTok{n.ahead=}\DecValTok{5}\NormalTok{,}\DataTypeTok{limits=}\NormalTok{F)}
\end{Highlighting}
\end{Shaded}

\includegraphics{SBramhall_Final_files/figure-latex/Prob 3a-Univaritate Analysis for Particulates-2.pdf}
\includegraphics{SBramhall_Final_files/figure-latex/Prob 3a-Univaritate Analysis for Particulates-3.pdf}

 Univariate Analysis for Particulates

Since the data is weekly, removing the weekly seasonality occurs first.
The differenced data is checked by viewing the resulting ACF with 95\%
confidence limits and results appear white. Next, the model is selected
using the AIC and BIC criteria and both result in a ARMA(2,1) model. The
Ljung-Box test with K=24 and K=48 both fail to reject the null and the
Box-Jenkins test all support white noise. Since the data is weekly
seasonal, an ARUMA forecast is performed and results are saved for later
use.

\begin{Shaded}
\begin{Highlighting}[]
\CommentTok{# -- Univariate Analysis of Temperatures Using ARUMA --}

\NormalTok{temp52 =}\StringTok{ }\KeywordTok{artrans.wge}\NormalTok{(data}\OperatorTok{$}\NormalTok{temp,}\KeywordTok{c}\NormalTok{(}\KeywordTok{rep}\NormalTok{(}\DecValTok{0}\NormalTok{,}\DecValTok{51}\NormalTok{),}\DecValTok{1}\NormalTok{))      }\CommentTok{# since we know the data is weekly, remove weekly trend}
\end{Highlighting}
\end{Shaded}

\includegraphics{SBramhall_Final_files/figure-latex/Prob 3a-Univaritate Analysis for Temperature-1.pdf}

\begin{Shaded}
\begin{Highlighting}[]
\KeywordTok{acf}\NormalTok{(temp52)                                         }\CommentTok{# confirm ACF is ~ white noise, it is!}

\CommentTok{# model selection}
\KeywordTok{aic5.wge}\NormalTok{(temp52)                                    }\CommentTok{# AIC picks ARMA(0,0)}
\end{Highlighting}
\end{Shaded}

\begin{verbatim}
## ---------WORKING... PLEASE WAIT... 
## 
## 
## Five Smallest Values of  aic
\end{verbatim}

\begin{verbatim}
##      p    q        aic
## 1    0    0   4.153692
## 6    1    2   4.156893
## 8    2    1   4.157374
## 4    1    0   4.158061
## 2    0    1   4.158063
\end{verbatim}

\begin{Shaded}
\begin{Highlighting}[]
\KeywordTok{aic5.wge}\NormalTok{(temp52,}\DataTypeTok{type =} \StringTok{"bic"}\NormalTok{)                       }\CommentTok{# BIC picks ARMA(0,0)}
\end{Highlighting}
\end{Shaded}

\begin{verbatim}
## ---------WORKING... PLEASE WAIT... 
## 
## 
## Five Smallest Values of  bic
\end{verbatim}

\begin{verbatim}
##      p    q        bic
## 1    0    0   4.162733
## 4    1    0   4.176142
## 2    0    1   4.176144
## 7    2    0   4.186350
## 3    0    2   4.186633
\end{verbatim}

\begin{Shaded}
\begin{Highlighting}[]
\CommentTok{# Check for white noise}
\KeywordTok{ljung.wge}\NormalTok{(temp52)                                   }\CommentTok{# FTR, p-value=.129}
\end{Highlighting}
\end{Shaded}

\begin{verbatim}
## Obs 0.004099555 0.05624436 -0.05326951 0.0573932 -0.04861631 0.03098495 0.02718822 0.1004989 0.08851697 -0.01558239 -0.01477092 -0.08380333 -0.01536697 -0.05652122 -0.01491832 0.07303893 0.07379244 0.0110402 -0.03145286 -0.02226394 0.02004529 -0.0838127 0.02392291 0.07002157
\end{verbatim}

\begin{verbatim}
## $test
## [1] "Ljung-Box test"
## 
## $K
## [1] 24
## 
## $chi.square
## [1] 31.90274
## 
## $df
## [1] 24
## 
## $pval
## [1] 0.1294229
\end{verbatim}

\begin{Shaded}
\begin{Highlighting}[]
\KeywordTok{ljung.wge}\NormalTok{(temp52, }\DataTypeTok{K =} \DecValTok{48}\NormalTok{)                           }\CommentTok{# Barely reject Ho, p-value=.032}
\end{Highlighting}
\end{Shaded}

\begin{verbatim}
## Obs 0.004099555 0.05624436 -0.05326951 0.0573932 -0.04861631 0.03098495 0.02718822 0.1004989 0.08851697 -0.01558239 -0.01477092 -0.08380333 -0.01536697 -0.05652122 -0.01491832 0.07303893 0.07379244 0.0110402 -0.03145286 -0.02226394 0.02004529 -0.0838127 0.02392291 0.07002157 0.03883168 0.004168604 0.005689672 -0.141479 -0.03792635 0.007452253 -0.03892399 -0.01262782 0.01859873 0.09041441 -0.06857444 -0.07895039 -0.02881891 0.06810352 -0.0138232 0.06763404 -0.06651562 0.07729948 -0.01803334 -0.0007800483 0.01314484 -0.06913632 0.03518071 -0.0288905
\end{verbatim}

\begin{verbatim}
## $test
## [1] "Ljung-Box test"
## 
## $K
## [1] 48
## 
## $chi.square
## [1] 67.73833
## 
## $df
## [1] 48
## 
## $pval
## [1] 0.0317127
\end{verbatim}

\begin{Shaded}
\begin{Highlighting}[]
\KeywordTok{acf}\NormalTok{(temp52,}\DataTypeTok{lag.max =} \DecValTok{48}\NormalTok{)                            }\CommentTok{# Box-Jenkins, acf looks consistent with white noise as previosly shown}

\CommentTok{# although Ljung0-Box with K=48 barely rejects Ho, K=24 FTR Ho and Box-Jenkins ACF is consisten with white noise, looks good}

\CommentTok{# Plot ARUMA Forecast}
\NormalTok{predsTemp =}\StringTok{ }\KeywordTok{fore.aruma.wge}\NormalTok{(data}\OperatorTok{$}\NormalTok{temp, }\DataTypeTok{s=}\DecValTok{52}\NormalTok{, }\DataTypeTok{n.ahead=}\DecValTok{5}\NormalTok{, }\DataTypeTok{limits=}\NormalTok{F)}
\end{Highlighting}
\end{Shaded}

\includegraphics{SBramhall_Final_files/figure-latex/Prob 3a-Univaritate Analysis for Temperature-2.pdf}
\includegraphics{SBramhall_Final_files/figure-latex/Prob 3a-Univaritate Analysis for Temperature-3.pdf}

 Univariate Analysis for Temperature

Since the data is weekly, removing the weekly seasonality occurs first.
The differenced data was checked by viewing the resulting ACF with 95\%
confidence limits and results appear white. Next, the model was selected
using the AIC and BIC criteria and both resulted in a ARMA(0,0) model.
The Ljung-Box test with K=24 failed to reject the null but the Ljung-Box
test with K=48 barely rejects Ho with a p-value=.032. However, the
Box-Jenkins test support white noise so no concerns. Since the data is
weekly seasonal, an ARUMA forecast is performed and results are saved
for later use.

\begin{Shaded}
\begin{Highlighting}[]
\CommentTok{# -- Model cmort based on predicted part, predicted temp, and Week using MLR with Correlated Errors --}

\NormalTok{ksfit =}\StringTok{ }\KeywordTok{lm}\NormalTok{(cmort}\OperatorTok{~}\NormalTok{temp}\OperatorTok{+}\NormalTok{part}\OperatorTok{+}\NormalTok{Week, }\DataTypeTok{data =}\NormalTok{ data)       }\CommentTok{# get linear fit to access residuals}

\CommentTok{# get model for residuals}
\NormalTok{phi =}\StringTok{ }\KeywordTok{aic.wge}\NormalTok{(ksfit}\OperatorTok{$}\NormalTok{residuals)                      }\CommentTok{# aic selects AR(2)}
\NormalTok{phi}
\end{Highlighting}
\end{Shaded}

\begin{verbatim}
## $type
## [1] "aic"
## 
## $value
## [1] 3.528269
## 
## $p
## [1] 2
## 
## $q
## [1] 0
## 
## $phi
## [1] 0.2477388 0.3581387
## 
## $theta
## [1] 0
## 
## $vara
## [1] 33.66498
\end{verbatim}

\begin{Shaded}
\begin{Highlighting}[]
\CommentTok{# fit arima with residual phis, remove weekly seasonality and incl ext vars temp, part, week}
\KeywordTok{attach}\NormalTok{(data)}
\end{Highlighting}
\end{Shaded}

\begin{verbatim}
## The following objects are masked from package:astsa:
## 
##     cmort, part
\end{verbatim}

\begin{Shaded}
\begin{Highlighting}[]
\NormalTok{fit =}\StringTok{ }\KeywordTok{arima}\NormalTok{(cmort,}\DataTypeTok{order=}\KeywordTok{c}\NormalTok{(phi}\OperatorTok{$}\NormalTok{p,}\DecValTok{0}\NormalTok{,}\DecValTok{0}\NormalTok{), }\DataTypeTok{seasonal=}\KeywordTok{list}\NormalTok{(}\DataTypeTok{order=}\KeywordTok{c}\NormalTok{(}\DecValTok{1}\NormalTok{,}\DecValTok{0}\NormalTok{,}\DecValTok{0}\NormalTok{),}\DataTypeTok{period=}\DecValTok{52}\NormalTok{), }\DataTypeTok{xreg =} \KeywordTok{cbind}\NormalTok{(temp, part, Week))}
\NormalTok{fit}
\end{Highlighting}
\end{Shaded}

\begin{verbatim}
## 
## Call:
## arima(x = cmort, order = c(phi$p, 0, 0), seasonal = list(order = c(1, 0, 0), 
##     period = 52), xreg = cbind(temp, part, Week))
## 
## Coefficients:
##          ar1     ar2   sar1  intercept    temp    part     Week
##       0.3909  0.4149  0.082    87.5975  0.0258  0.1407  -0.0292
## s.e.  0.0441  0.0404  0.049     4.0150  0.0530  0.0290   0.0086
## 
## sigma^2 estimated as 28.81:  log likelihood = -1575.14,  aic = 3166.29
\end{verbatim}

\begin{Shaded}
\begin{Highlighting}[]
\CommentTok{# The intercept is significantly different from 0, the particulates and Time are significant, but the temperature is not.}

\CommentTok{# Check residuals for white noise}
\KeywordTok{ljung.wge}\NormalTok{(fit}\OperatorTok{$}\NormalTok{residuals)                            }\CommentTok{# Barely reject, p-value=.048}
\end{Highlighting}
\end{Shaded}

\begin{verbatim}
## Obs -0.008773997 -0.006778 0.02847037 0.001592821 0.04173621 0.03014614 -0.02519791 -0.03614844 0.08896799 -0.04873154 0.003349583 0.04272511 -0.05378803 -0.03121467 0.1104465 -0.0511791 -0.05781976 -0.05203572 -0.1226383 -0.01661192 0.03274971 -0.06125045 -0.08319911 0.009083812
\end{verbatim}

\begin{verbatim}
## $test
## [1] "Ljung-Box test"
## 
## $K
## [1] 24
## 
## $chi.square
## [1] 36.59418
## 
## $df
## [1] 24
## 
## $pval
## [1] 0.04801085
\end{verbatim}

\begin{Shaded}
\begin{Highlighting}[]
\KeywordTok{ljung.wge}\NormalTok{(fit}\OperatorTok{$}\NormalTok{residuals, }\DataTypeTok{K =} \DecValTok{48}\NormalTok{)                    }\CommentTok{# Reject Ho, p-value=.003}
\end{Highlighting}
\end{Shaded}

\begin{verbatim}
## Obs -0.008773997 -0.006778 0.02847037 0.001592821 0.04173621 0.03014614 -0.02519791 -0.03614844 0.08896799 -0.04873154 0.003349583 0.04272511 -0.05378803 -0.03121467 0.1104465 -0.0511791 -0.05781976 -0.05203572 -0.1226383 -0.01661192 0.03274971 -0.06125045 -0.08319911 0.009083812 -0.06018696 -0.05789479 0.03458024 -0.1371038 -0.02600092 -0.07277285 -0.03981533 -0.03157352 0.009442916 -0.06678764 -0.05777429 -0.004599118 -0.01492029 -0.03032822 -0.004028157 -0.06378089 0.01029915 0.03886128 -0.006943152 0.04753463 0.04475963 0.0230044 0.1033364 0.110844
\end{verbatim}

\begin{verbatim}
## $test
## [1] "Ljung-Box test"
## 
## $K
## [1] 48
## 
## $chi.square
## [1] 79.51156
## 
## $df
## [1] 48
## 
## $pval
## [1] 0.002852882
\end{verbatim}

\begin{Shaded}
\begin{Highlighting}[]
\KeywordTok{acf}\NormalTok{(fit}\OperatorTok{$}\NormalTok{residuals,}\DataTypeTok{lag.max =} \DecValTok{48}\NormalTok{)                     }\CommentTok{# Box-Jenkins, acf looks}
\end{Highlighting}
\end{Shaded}

\includegraphics{SBramhall_Final_files/figure-latex/Prob 3a-Cmort MLR with Correlated Errors-1.pdf}

\begin{Shaded}
\begin{Highlighting}[]
\CommentTok{# acf show maybe 3 points out of 40 outside the limits but this really isn't bad}

\CommentTok{# build DF of predicted variables}
\NormalTok{last30 =}\StringTok{ }\KeywordTok{data.frame}\NormalTok{(}\DataTypeTok{temp =}\NormalTok{ data}\OperatorTok{$}\NormalTok{temp[}\DecValTok{479}\OperatorTok{:}\DecValTok{508}\NormalTok{], }\DataTypeTok{part =}\NormalTok{ data}\OperatorTok{$}\NormalTok{part[}\DecValTok{479}\OperatorTok{:}\DecValTok{508}\NormalTok{], }\DataTypeTok{Week =} \KeywordTok{seq}\NormalTok{(}\DecValTok{479}\NormalTok{,}\DecValTok{508}\NormalTok{,}\DecValTok{1}\NormalTok{))}

\CommentTok{# get predictions}
\NormalTok{predsCMort =}\StringTok{ }\KeywordTok{predict}\NormalTok{(fit,}\DataTypeTok{newxreg=}\NormalTok{last30)}

\CommentTok{# plot the next 30 cmorts}
\KeywordTok{plot}\NormalTok{(}\KeywordTok{seq}\NormalTok{(}\DecValTok{1}\NormalTok{,}\DecValTok{508}\NormalTok{,}\DecValTok{1}\NormalTok{),cmort,}\DataTypeTok{type=}\StringTok{"l"}\NormalTok{,}\DataTypeTok{xlim=}\KeywordTok{c}\NormalTok{(}\DecValTok{0}\NormalTok{,}\DecValTok{538}\NormalTok{),}\DataTypeTok{ylab=}\StringTok{"Cariac Mortality"}\NormalTok{,}\DataTypeTok{main=}\StringTok{"30 Week Cardiac Mortality Forecast"}\NormalTok{)}
\KeywordTok{lines}\NormalTok{(}\KeywordTok{seq}\NormalTok{(}\DecValTok{509}\NormalTok{,}\DecValTok{538}\NormalTok{,}\DecValTok{1}\NormalTok{),predsCMort}\OperatorTok{$}\NormalTok{pred,}\DataTypeTok{type=}\StringTok{"l"}\NormalTok{,}\DataTypeTok{col=}\StringTok{"blue"}\NormalTok{)}
\end{Highlighting}
\end{Shaded}

\includegraphics{SBramhall_Final_files/figure-latex/Prob 3a-Cmort MLR with Correlated Errors-2.pdf}

\begin{Shaded}
\begin{Highlighting}[]
\NormalTok{ASE =}\StringTok{ }\KeywordTok{mean}\NormalTok{((data}\OperatorTok{$}\NormalTok{cmort[}\DecValTok{479}\OperatorTok{:}\DecValTok{508}\NormalTok{] }\OperatorTok{-}\StringTok{ }\NormalTok{predsCMort}\OperatorTok{$}\NormalTok{pred)}\OperatorTok{^}\DecValTok{2}\NormalTok{)}
\NormalTok{ASE}
\end{Highlighting}
\end{Shaded}

\begin{verbatim}
## [1] 56.59301
\end{verbatim}

 Cmort Multiple Linear Regression with Correlated Errors

The goal is to develop a model to predict cardiac mortality. Since the
variables were seen earlier to be independent a MLR model using
correlated errors was chosen. A linear fit was performed using
particulates, temperature, and time. AIC selecetd an AR(2) model for the
residuals. The ARIMA function was used to fit the residuals, applying
weekly seasonality and including external varibles temperature,
particulate, and time (week \#). The fitted results showed the intercept
is significantly different from 0, the particulates and time are
significant, but the temperature is not.

Next, the residuals were checked for white noise. The Ljung-Box test
with K=24 barely rejects with a p-value=.048 and with K=48 it rejects
Ho. When reviewing the ACFs, maybe 3 points our of 40 are outside the
limits but this is not bad. Judgement is to proceed.

A dataframe of the predicted temperature and particulates from the
univariate analysis along with the week number was created. The plot
above shows the realization with the predicted values in blue. The ASE
for this model is \textbf{56.59}.

\subsubsection{3b. Compare the models and describe which univariate
model you feel is the best and
why.}\label{b.-compare-the-models-and-describe-which-univariate-model-you-feel-is-the-best-and-why.}

\subsubsection{4a. Perform a multivariate analysis using at least a VAR
or MLR with correlated errors and a MLP model. Clearly explain how you
arrived at the final model. Use forecasted values of the predictors
where
appropriate.}\label{a.-perform-a-multivariate-analysis-using-at-least-a-var-or-mlr-with-correlated-errors-and-a-mlp-model.-clearly-explain-how-you-arrived-at-the-final-model.-use-forecasted-values-of-the-predictors-where-appropriate.}

\subsubsection{4b. Fit and evaluate an ensemble model from the models
you fit in
4a.}\label{b.-fit-and-evaluate-an-ensemble-model-from-the-models-you-fit-in-4a.}

\subsubsection{4c. Compare these models and describe which multivariate
model you feel is the best and
why.}\label{c.-compare-these-models-and-describe-which-multivariate-model-you-feel-is-the-best-and-why.}

\subsubsection{5. Use the model you feel is most useful to forecat the
next 5 weeks of respiratory
mortality.}\label{use-the-model-you-feel-is-most-useful-to-forecat-the-next-5-weeks-of-respiratory-mortality.}


\end{document}
